\documentclass[12pt]{article}


\title{term paper  topic - Blood bank management system}
\author{Shraddha Tandan ,Roll no 19111055}
\date{January 2022}

\begin{document}

\maketitle

\section{abstract }
\section{introduction}
\section{proposed system}
\section*{module: 1 Android application
2. Hospital web application 
3. Blood bank web application 
Database }
\section{Results}
\section{conclusion}
\section{references}
\end{document}
\documentclass[12pt]{article}
\usepackage[utf8]{inputenc}

\title{Update No   2}
\author{Shraddha  tandan ,roll no 19111055}
\date{January 2022}

\begin{document}

\maketitle

\section{abstract}— This paper presents a high-end system to bridge the gap between the blood donors and the people in need for
blood. Application for Blood Bank Management System is a way to synchronize Blood banks and Hospitals with the help of
Internet. It is a Web Application through which Registered Hospitals can check the availability of required Blood and can
send Request for blood to the nearest blood bank or donor matching with blood requirement and can be ordered online as
and when required. Blood bank can also send a request to another blood bank for unavailable blood. Person willing to donate
blood can find out nearest blood banks using Blood Bank Management Android Application. The location of the blood bank
can also be traced using maps. The Android application can be accessed only by the donors to search the blood donation
centers and the requesting blood banks and hospitals to search the nearest blood banks and donors.
\section*{Index Terms}— Blood Bank Management, Blood Bank, Hospital, Donor, Recipient

\end{document}
\documentclass[12pt]{article}
\usepackage[utf8]{inputenc}
\title{update 4}
\author{Shraddha Tandan Roll no  19111055}
\date{February 2022}

\begin{document}

\maketitle

\section*{Introduction}. INTRODUCTION
The population of the world is multiplying with each
coming year and so are the diseases and health issues.
With an increase in the population there is an increase
in the need of blood. The growing population of the
world results in a lot of potential blood donors. But in
spite of this not more than 10% of the total world
population participates in blood donation. With the
growing population and the advancement in medical
science the demand for blood has also increased. Due
to the lack of communication between the blood
donors and the blood recipients, most of the patients
in need of blood do not get blood on time and hence
lose their lives. There is a dire need of
synchronization between the blood donors and
hospitals and the blood banks. This improper
management of blood leads to wastage of the
available blood inventory. Improper communication
and synchronization between the blood banks and
hospitals leads to wastage of the blood available.
These problems can be dealt with by automating the
existing manual blood bank management system. A
high-end, efficient, highly available and scalable
system has to be developed to bridge the gap between
the donors and the recipients and to reduce the efforts
required to search for blood donors. 

\end{document}

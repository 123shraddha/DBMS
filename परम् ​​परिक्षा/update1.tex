\documentclass[12pt]{article}


\title{term paper  topic - Blood bank management system}
\author{Shraddha Tandan ,Roll no 19111055}
\date{January 2022}

\begin{document}

\maketitle

\section{abstract }
\section{introduction}
\section{proposed system}
\section*{module: 1 Android application
2. Hospital web application 
3. Blood bank web application 
Database }
\section{Results}
\section{conclusion}
\section{references}
\end{document}
\documentclass[12pt]{article}
\usepackage[utf8]{inputenc}

\title{Update No   2}
\author{Shraddha  tandan ,roll no 19111055}
\date{January 2022}

\begin{document}

\maketitle

\section{abstract}— This paper presents a high-end system to bridge the gap between the blood donors and the people in need for
blood. Application for Blood Bank Management System is a way to synchronize Blood banks and Hospitals with the help of
Internet. It is a Web Application through which Registered Hospitals can check the availability of required Blood and can
send Request for blood to the nearest blood bank or donor matching with blood requirement and can be ordered online as
and when required. Blood bank can also send a request to another blood bank for unavailable blood. Person willing to donate
blood can find out nearest blood banks using Blood Bank Management Android Application. The location of the blood bank
can also be traced using maps. The Android application can be accessed only by the donors to search the blood donation
centers and the requesting blood banks and hospitals to search the nearest blood banks and donors.
\section*{Index Terms}— Blood Bank Management, Blood Bank, Hospital, Donor, Recipient

\end{document}
\documentclass[12pt]{article}
\usepackage[utf8]{inputenc}
\title{update 4}
\author{Shraddha Tandan Roll no  19111055}
\date{February 2022}

\begin{document}

\maketitle

\section*{Introduction}. INTRODUCTION
The population of the world is multiplying with each
coming year and so are the diseases and health issues.
With an increase in the population there is an increase
in the need of blood. The growing population of the
world results in a lot of potential blood donors. But in
spite of this not more than 10% of the total world
population participates in blood donation. With the
growing population and the advancement in medical
science the demand for blood has also increased. Due
to the lack of communication between the blood
donors and the blood recipients, most of the patients
in need of blood do not get blood on time and hence
lose their lives. There is a dire need of
synchronization between the blood donors and
hospitals and the blood banks. This improper
management of blood leads to wastage of the
available blood inventory. Improper communication
and synchronization between the blood banks and
hospitals leads to wastage of the blood available.
These problems can be dealt with by automating the
existing manual blood bank management system. A
high-end, efficient, highly available and scalable
system has to be developed to bridge the gap between
the donors and the recipients and to reduce the efforts
required to search for blood donors. 

\end{document}
\documentclass[12pt]{article}
\usepackage[utf8]{inputenc}

\title{update 5}
\author{Shraddha  Tandan   Roll no 19111055}
\date{February 2022}

\begin{document}

\maketitle

\section*{Functions provided by a database management
system }Database managers all provide the following functions:

\section*{1. Data entry.} There must be a screen that allows you
to enter, edit, alter, add, or delete data items.
\section*{2. Sorting capability.}
 This feature allows you to arrange file records in different ways, e.g., alphabetically
or chronologically
\section*{3. Search or find capacity }The search function
allows you to find and retrieve records that match specified data in one or more fields.
\section*{4. Report production.} This capability is very important to most blood bankers; it enables you to print neat
groups and subfiles of information.
\section*{5. Calculation and graphics capacities}. These are
included in some database managers such as Reflex
\section*{6. Programming and creating macros.} Programming languages built into some database managers
permit the user to develop macros. These make it possible to execute a long series of formatting or calculating instructions with a few keystrokes.
\section*{7. Relating files}. Relational database management
systems such as dBase can automatically relate fields in
one file to specified fields in other files.


\end{document}
\documentclass[12pt]{article}
\usepackage[utf8]{inputenc}

\title{update 6}
\author{Shraddha Tandan   Roll no 19111055}
\date{February 2022}

\begin{document}

\maketitle

\section*{PROPOSED SYSTEM}The proposed system (Blood Bank Management
System) is designed to help the Blood Bank
administrator to meet the demand of Blood by
sending and/or serving the request for Blood as and
when required.The proposed system gives the
procedural approach of how to bridge the gap
between Recipient, Donor, and Blood Banks. This
Application will provide a common ground for all the
three parties (i.e. Recipient, Donor, and Blood Banks)and will ensure the fulfillment of demand for Blood
requested by Recipient and/or Blood Bank.
The proposed system consists of the following goals
and has the scope as follows:a) Goals:

● To ease the process of blood donation and
reception.

● To improve the existing system.

● To develop a scalable system.
● To be highly available

b) Scope:

● Ensure that all the functionalities of a
manual blood bank are covered

● To include all the blood banks at least within
a city.

● Make sure the program is simple and easy to
use.


\end{document}\documentclass[12pt]{article}
\usepackage[utf8]{inputenc}

\title{UPDATE    7}
\author{Shraddha Tandan   Roll no 19111055}
\date{April 2022}

\begin{document}

\maketitle

\section*{Module1 }[ Android application]
This module consists of the process of how blood
donation process is done in this system. The blood
donor can find out the nearest blood banks available
according to his/her current location based on the
GPRS feature used in this system. The blood donor
will then have to register themselves on the
application for validation purpose and further
donating the blood to a particular blood bank. These
blood donors can later also be contacted based on the
availability status they have updated on the system
for further contact in case of requirement of blood of
their blood group.
\section*{module 2}[Hospital Web Application]
This module consists of the process of how recipients
are going to request for the required amount of blood
from the blood bank. The recipient has to make use of
unique hospital id which is registered in the hospital’s
database .Only those requests made through a valid
hospital id will be considered as valid requests. While
requesting for the required amount of blood , the

\end{document}
\documentclass[12PT]{article}
\usepackage[utf8]{inputenc}

\title{UPDATE 8}
\author{Shraddha Tandan Roll No 19111055}
\date{April 2022}

\begin{document}

\maketitle

\section*{Module 3}: {Blood Bank Web Application}
This module consists of the process of how the
requests from recipients for the required blood are
served. The Blood Bank first checks whether the
request is a valid one. After validation it checks the
hospital’s database to ensure that the required amount
of blood is not available in that hospital and after the
request is served. The blood bank module also
consists of requesting the blood when urgently
needed from other banks and from the registered
donors who have kept their status as available for
further contact.
\section*{Module 4}:{ Database}
Separate databases are maintained for the android
application which mainly consists of the registered
donor’s information, the database of hospital web
application which consists of the records of available
blood group samples and also the database of the
blood bank web application which consists of records
of the blood group samples and their respective
quantity available in every blood bank. The database
of the android application is in sync with the database
of the web application.All the databases will be
hosted on the cloud server . This will make them
more reliable and also will make them scalable.

\end{document}
u
